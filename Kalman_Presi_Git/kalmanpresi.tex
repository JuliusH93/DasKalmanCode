\documentclass[aspectratio=169,babel]{beamer}

\usetheme{boxes}

\usepackage[utf8]{inputenc}
\usepackage[T1]{fontenc}
\usepackage{bm}        % standard math notation (fonts)
\usepackage{amsfonts}
\usepackage{amssymb}
\usepackage{amsmath}  % standard math notation (vectors/sets/...)
\usepackage[ngerman]{babel}
\usepackage[babel,german=quotes]{csquotes}
%\usepackage{listings}
\usepackage{graphicx}
\usepackage{ulem}
\usepackage{hyperref}
\usepackage{graphicx} % eps graphics support
\usepackage{epsfig}
\usepackage{subfigure}
\usepackage{times}           % scalable fonts
\usepackage{docmute}

\newcommand{\qq}[1]{\glqq #1\grqq{}}

% Adds titles for each section using section name as title
\AtBeginSection[]{
  \begin{frame}
  \vfill
  \centering
  \begin{beamercolorbox}[sep=8pt,center,shadow=true,rounded=true]{title}
    \usebeamerfont{title}\insertsectionhead\par%
  \end{beamercolorbox}
  \vfill
  \end{frame}
}

\setbeamertemplate{footline}[page number]
\setbeamertemplate{section in toc}{\inserttocsectionnumber.~\inserttocsection}
%\setbeamertemplate{frametitle}[default][center]

\beamertemplatenavigationsymbolsempty

\title{Das Kalman Filter}
\author{Markus Bullmann, Julius Hackel, \\ Stefan Gerasch, Felix Keller}
\date{\today}

\begin{document}
    \maketitle
    \begin{frame}
			\frametitle{Gliederung}
			\tableofcontents
		\end{frame}
    
    \section{Einführung}

\begin{frame}
    \frametitle{Einführung}
		
    Lorem ipsum dolor sit amet, consectetuer adipiscing elit. Maecenas porttitor congue massa. Fusce posuere, magna sed pulvinar ultricies, purus lectus malesuada libero, sit amet commodo magna eros quis urna.
    Nunc viverra imperdiet enim. Fusce est. Vivamus a tellus.
    Pellentesque habitant morbi tristique senectus et netus et malesuada fames ac turpis egestas. Proin pharetra nonummy pede. Mauris et orci.
    Aenean nec lorem. In porttitor. Donec laoreet nonummy augue.
    Suspendisse dui purus, scelerisque at, vulputate vitae, pretium mattis, nunc. Mauris eget neque at sem venenatis eleifend. Ut nonummy.      
\end{frame}
    \section{Einfaches Beispiel, Eindimensional}

\begin{frame}
    \frametitle{Einfaches Beispiel, Eindimensional}
		
    Lorem ipsum dolor sit amet, consectetuer adipiscing elit. Maecenas porttitor congue massa. Fusce posuere, magna sed pulvinar ultricies, purus lectus malesuada libero, sit amet commodo magna eros quis urna.
    Nunc viverra imperdiet enim. Fusce est. Vivamus a tellus.
    Pellentesque habitant morbi tristique senectus et netus et malesuada fames ac turpis egestas. Proin pharetra nonummy pede. Mauris et orci.
    Aenean nec lorem. In porttitor. Donec laoreet nonummy augue.
    Suspendisse dui purus, scelerisque at, vulputate vitae, pretium mattis, nunc. Mauris eget neque at sem venenatis eleifend. Ut nonummy.      
\end{frame}
    \section{Komplexe Matrixform des Kalman Filter}

\begin{frame}
    \frametitle{Komplexe Matrixform des Kalman Filter}
		
    Lorem ipsum dolor sit amet, consectetuer adipiscing elit. Maecenas porttitor congue massa. Fusce posuere, magna sed pulvinar ultricies, purus lectus malesuada libero, sit amet commodo magna eros quis urna.
    Nunc viverra imperdiet enim. Fusce est. Vivamus a tellus.
    Pellentesque habitant morbi tristique senectus et netus et malesuada fames ac turpis egestas. Proin pharetra nonummy pede. Mauris et orci.
    Aenean nec lorem. In porttitor. Donec laoreet nonummy augue.
    Suspendisse dui purus, scelerisque at, vulputate vitae, pretium mattis, nunc. Mauris eget neque at sem venenatis eleifend. Ut nonummy.      
\end{frame}
    \section{Herleitung der Formeln}

\begin{frame}
    \frametitle{Herleitung der Formeln}
		
    Lorem ipsum dolor sit amet, consectetuer adipiscing elit. Maecenas porttitor congue massa. Fusce posuere, magna sed pulvinar ultricies, purus lectus malesuada libero, sit amet commodo magna eros quis urna.
    Nunc viverra imperdiet enim. Fusce est. Vivamus a tellus.
    Pellentesque habitant morbi tristique senectus et netus et malesuada fames ac turpis egestas. Proin pharetra nonummy pede. Mauris et orci.
    Aenean nec lorem. In porttitor. Donec laoreet nonummy augue.
    Suspendisse dui purus, scelerisque at, vulputate vitae, pretium mattis, nunc. Mauris eget neque at sem venenatis eleifend. Ut nonummy.      
\end{frame}
    \section{Beispiele, Fazit}

\begin{frame}
    \frametitle{Beispiele, Fazit}
		
    Lorem ipsum dolor sit amet, consectetuer adipiscing elit. Maecenas porttitor congue massa. Fusce posuere, magna sed pulvinar ultricies, purus lectus malesuada libero, sit amet commodo magna eros quis urna.
    Nunc viverra imperdiet enim. Fusce est. Vivamus a tellus.
    Pellentesque habitant morbi tristique senectus et netus et malesuada fames ac turpis egestas. Proin pharetra nonummy pede. Mauris et orci.
    Aenean nec lorem. In porttitor. Donec laoreet nonummy augue.
    Suspendisse dui purus, scelerisque at, vulputate vitae, pretium mattis, nunc. Mauris eget neque at sem venenatis eleifend. Ut nonummy.      
\end{frame}
\end{document}